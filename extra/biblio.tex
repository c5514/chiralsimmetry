\cleardoublepage
\addcontentsline{toc}{chapter}{Bibliografía}
\begin{thebibliography}{X}
	%\bibitem{Lorentz1916} \textsc{H.A. Lorentz}.
	%\textit{The Thoery of Electrons, and its applications to the phenomena of light and radiant heat}. Segunda edici\'on.
	%B.G. Teubner Verlag, Alemania, Leipzig (1916).

	% \bibitem{Einstein1985} \textsc{A. Einstein}.
	% \textit{Textos recogidos de Sobre la teor\'ia de la relatividad y otras aportaciones cient\'ificas, Mis ideas y opiniones}. Sarpe, Madrid.  Traducción de José M. Álvarez Flores y Ana Goldar. (1985)
	%
	% \bibitem{James2012} \textsc{James D. Wells}.
	% \textit{Effective Theories in Physics. From Planetary Orbits to Elementary Particle Masses}. Springer.  (2012)
	%
	% \bibitem{Einstein1905} \textsc{A. Einstein}.
	% \textit{Zur elektrodynamik bewegter körper}.  Annalen der physik, 322(10), 891-921. (1905)
	%
	% \bibitem{Einstein1907} \textsc{A. Einstein}.
	% \textit{On the relativity principle and the conclusions drawn from it}.  Jahrbuch der Radioaktivität und Elektronik, 4, 411-462. (1907)
	%
	% \bibitem{d'Inverno1992} \textsc{R. d'Inverno}.
	% \textit{Introducing Einstein's relativity}. Oxford University Press Inc. N.Y., EE.UU. (1992)
	%
	% \bibitem{Barut1980} \textsc{A. Barut}.
	% \textit{Electrodynamics and Classical Theory of Fields and Particles}. Dover Publications, Inc. N.Y., EE.UU. (1980)

	\bibitem{Jain2022} Jain, V. K. (2022). Solid State Physics. Springer International Publishing.
	\bibitem{Lomer1955} Lomer, W. M. (1955). The valence bands in two-dimensional graphite. Proceedings of the Royal Society of London. Series A. Mathematical and Physical Sciences, 227(1170), 330–349.
	\bibitem{Bradley2009}Bradley, C. J., \& Cracknell, A. P. (2009). The Mathematical Theory Of Symmetry In Solids: Representation theory for point groups and space groups. Oxford University Press.
	\bibitem{Seitz1936}Seitz, F. (1936). On the Reduction of Space Groups. Annals of Mathematics, 37(1), 17–28.
	\bibitem{Schonenberger2000}Schonenberger, C. (2000). Bandstructure of Graphene and Carbon Nanotubes: An Exercise in Condensed Matter Physics.
	\bibitem{Mervyn2000}Mervyn, R. (2015). The Tight Binding Method.
	\bibitem{Bruss2004}	Bruus, H., \& Flensberg, K. (2004). Many–Body Quantum Theory in Condensed Matter Physics: An Introduction. Oxford University Press.
	\bibitem{Fuchs2008}Fuchs, J.-N., \& Goerbig, M. O. (2008). Introduction to the Physical Properties of Graphene.
	\bibitem{Bernevig2013}Bernevig, B. A., \& Hughes, T. L. (2013). Topological Insulators and Topological Superconductors. Princeton University Press.
	\bibitem{Riazuddin2012}Riazuddin. (2012). Dirac equation for quasi-particles in graphene and quantum field theory of their Coulomb interaction. International Journal of Modern Physics B, 26(21), 1242005.
	\bibitem{Sadurni2014} Sadurní, E., Rivera-Mociños, E., \& Rosado, A. (2014). Discrete symmetry in graphene: The Dirac equation and beyond. [Online]. Available:\url{https://arxiv.org/pdf/1411.6319}
	\bibitem{Jena2022}D. Jena. (2022). Quantum Physics of Semiconductor Materials and Devices. Oxford University Press.
	\bibitem{Hamermesh1989}Hamermesh, M. (1989). Group Theory and Its Application to Physical Problems. Dover Publications.
\end{thebibliography}
%\nocite{Lorentz1916}
%\nocite{Jackson1999}
%\nocite{Einstein1985}
%\nocite{Soto2016}
%\nocite{Greiner1996}
%\bibliographystyle{unsrt} % estilo de la bibliografía.
%\bibliography{biblio} % biblio.bib es el fichero donde está salvada la bibliografía.
